\chapter{Log analyzer}

\section{Useful shell commands}

\begin{lstlisting}[language=bash]
  $ grep -E "^Time =|Cour" log.run | less
\end{lstlisting}

\begin{lstlisting}[language=bash]
  $ grep -E "^Time =|p_rgh|PIMPLE" log.run | less
\end{lstlisting}

\section{fsAwk}

You first need to make symbolic links in /bin to python scripts. To do this:

\begin{lstlisting}[]
  $ cd ~/bin
  $ ln -s ~/dvt/foamBazar/pythonScripts/fsLog.awk .
  $ ln -s ~/dvt/foamBazar/pythonScripts/fsPlot.py . 
\end{lstlisting}

When typing \textit{fsPlot.py} and relative input,it will automatically parse the log file through the script "fsAwk" (if it is not done already). All the possible commands of \textit{fsPlot.py} are listed through the help command.
\begin{lstlisting}[language=bash]
  $ fsPlot.py --help
\end{lstlisting}

Below are some examples, based on a log file called "\textit{log.run}".
\begin{itemize}
\item Plot the time history of all the different Courant numbers (max and mean of flow and interface Courant numbers):
\begin{lstlisting}[language=bash]
  $ fsPlot.py -p co log.run
\end{lstlisting}

\item Plot the time history of maximum of flow Courant number:
\begin{lstlisting}[language=bash]
  $ fsPlot.py -p co -w max log.run
\end{lstlisting}

\item Plot the time history of maximum of interface Courant number:
\begin{lstlisting}[language=bash]
  $ fsPlot.py -p co -w max,interface log.run
\end{lstlisting}

\item Plot the time history of the initial residual of Ux:
\begin{lstlisting}[language=bash]
  $ fsPlot.py -p res -w init,Ux log.run
\end{lstlisting}

\item Plot the time history of the force in the z-direction at the final iteration:
\begin{lstlisting}[language=bash]
  $ fsPlot.py -p f -w z log.run
\end{lstlisting}

\item Plot the time history of the force in the z-direction at all iterations:
\begin{lstlisting}[language=bash]
  $ fsPlot.py -p f -w z -i
\end{lstlisting}

\item Plot the time history of the force in the z-direction at all iterations with markers:
\begin{lstlisting}[language=bash]
  $ fsPlot.py -p f -w z -i -l s:-,m:x
\end{lstlisting}

\end{itemize}
