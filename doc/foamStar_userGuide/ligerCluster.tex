\chapter{Using Liger cluster}


\section{Running a batch}

A batch "run.sh" for foamStar looks like this:
\begin{lstlisting}[language=bash]
#!/bin/bash -l
#SBATCH -J fs-KCS-pureEuler

# 5 hour wall-clock 
#SBATCH --account I1608251
#SBATCH -t 3-00:00:00
#SBATCH -n 72
#SBATCH -o log.run-%j

module load gcc/4.9.3 openmpi/1.8.4-gcc lapack/3.6.1/gcc/4.9.3
export FOAM_INST_DIR=/data/I1608251/OpenFOAM;
source /data/I1608251/OpenFOAM/OpenFOAM-2.4.x/etc/bashrc;
export LC_ALL=C

mpirun foamStar -parallel
\end{lstlisting}
You have to make sure that you change the name of the job (here "fs-KCS-pureEuler") and the number of processors (here 72) accordingly to your simulation.

To run the batch, the command is:
\begin{lstlisting}[language=bash]
  $ sbatch run.sh
\end{lstlisting}


To show the queue:
\begin{lstlisting}[language=bash]
  $ squeue
\end{lstlisting}

And to show only your queue (with your own username):
\begin{lstlisting}[language=bash]
  $ squeue | grep userName
\end{lstlisting}


To cancel the job, the command is:
\begin{lstlisting}[language=bash]
  $ scancel jobID
\end{lstlisting}


\section{Keeping alive}

You will get disconnected if you are inactive for more than 5 minutes. In order no to lose your connection (your location and your history - note: you can keep your history with screen), you can use a small trick by defining a keepAlive function in your .bashrc.


\begin{lstlisting}[language=bash]
alias keepAlive='while [ true ]; do clc; date; i=$[(RANDOM % 100)+10]; sleep $i; done'
\end{lstlisting}

You will have to launch "keepAlive" every time you are not working on your session.