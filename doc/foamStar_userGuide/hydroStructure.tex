\chapter{HydroStructure computations}
\label{homer}

It is possible to perform either quasi-static or elastic hydrostructure computations with foamStar. Quasi-static means one-way coupling with load transfer to FEM. Elastic computations take into account deformation of the structure and rely on mode shapes previously computed. Elastic computations offer a two-way coupling. It is possible to perform load transfer in addition to elastic computation.

In FoamStar, both methods rely on the use of HOMER as interface between the structural solver and the CFD computation.


\section{Elastic computations}

All structural information needed for mechanical equation solving and flexible modes can be pre-computed from an Homer 2 model.

\subsection{Running Homer}
Once the model has been set up, HmFEM has to be run with the following body options. Please refer to Homer user guide for more information about Homer keywords.

\begin{lstlisting}[language=fortran]
# Strcutural mesh information
STRUCTURALMESH = body.dat
STRUCTURALCOORDINATES = 0. 0. -11.75 0. 0. 0.
STRUCTURALLENGTHUNIT = m
STRUCTURALMASSUNIT = kg

# Number of flexible modes if required
STRUCTURALNFLEXIBLEMODES = 3

# Enable specific outputs for OpenFoam
PREPAREOPENFOAM = 1

# Sections definition
SECTIONS
CUT   -6.00 0. 12.015 1. 0. 0.
CUT   82.83 0. 12.015 1. 0. 0.
CUT  150.78 0. 12.015 1. 0. 0.
CUT  223.23 0. 12.015 1. 0. 0.
CUT  295.00 0. 12.015 1. 0. 0.
ENDSECTIONS
\end{lstlisting}

HmFEM will then run normally and create the following files that are useful for foamStar:
\begin{itemize}
\item mesh : Finite\_Elements\_Analysis{\textbackslash}HmFEM{\textbackslash}\textbf{body\_Bulk.dat}
\item mass matrix : Finite\_Elements\_Analysis{\textbackslash}HmFEM{\textbackslash}\textbf{body\_dmig.pch}
\item mode shapes : Finite\_Elements\_Analysis{\textbackslash}HmFEM{\textbackslash}\textbf{body\_md.pch}
\item don file : User\_Outputs{\textbackslash}HmFEM{\textbackslash}\textbf{body.don}
\end{itemize}

\subsection{Initialize structural information for foamStar}

Structural data pre-processing for foamStar is performed using the module \emph{initFlx} together with a standard OpenFoam input file \emph{initFlxDict} described hereafter:

\begin{lstlisting}[language=C]
FEM_STRUCTURALMESH_VTU
{
    datFile "body_Bulk.dat";
    mdFile "body_md.pch"; selected (7 8 9);
    dmigMfile "body_dmig.pch";
    dmigKfile "body_dmig.pch";
    pchCoordinate (0 0 -11.75 0 0 0);
    pchScaleMode  0.25909281809220411E+05;
    pchLengthUnit 1;
    pchMassUnit   1;

    patches (ship); ySym (true);
}
\end{lstlisting}

\begin{itemize}
\item \textbf{datFile} : finite element mesh provided by Homer (*\_Bulk.dat)
\item \textbf{mdFile} : modes shapes punch file given by Homer (*\_md.pch)
\item \textbf{selected} : select flexible modes to use for simulation (starting from mode 7)
\item \textbf{dmigMfile} : mass matrix punch file provided by Homer (*\_dmig.pch)
\item \textbf{dmigKfile} : stiffness matrix punch file provided by Homer (*\_dmig.pch)
\item \textbf{pchCoordinate} : structural coordinates used in Homer
\item \textbf{pchScaleMode} : value of scaling factor of first mode in Homer
\item \textbf{pchLengthUnit} : length scaling factor to meters (m=1, mm=0.001)
\item \textbf{pchMassUnit} : mass scaling factor to kilograms (kg=1, t=1000)
\item \textbf{pchMassUnit} : mass scaling factor to kilograms (kg=1, t=1000)
\item \textbf{patches} : define patch representing the hull
\item \textbf{patches} : define if Y-symmetry is used or not
\end{itemize}

foamStar module can then be launched with the following command:
\begin{lstlisting}[language=bash]
$ initFlx
\end{lstlisting}


\section{Load transfer to FEM}

What is important is to have proper equilibrium in structural computation. The algorithm is the following one:
\begin{itemize}
\item Building or getting the FEM model

\item Create a stl file from the skin of the structural model

\item Run HOMER on this model: first run hmFEM. Then proceed to balancing with hmSWB. In the input of hmSWB, specify:
\begin{lstlisting}[language=fortran]
USEOPENFOAMCASES = 1
SHIFTOPENFOAM = 0.1
\end{lstlisting}

The keyword "SHIFTOPENFOAM" controls the distance from which the Gauss points will be shifted towards the fluid domain. It is recommended to use 0.1m for a full scale ship (otherwise discretization of the geometry in the  CFD meshing may lead to points not located inside the domain).

\item In the output file "hmSWB.out", the result of the balancing looks like this:
\begin{lstlisting}[language=fortran]
Definition of structural model coordinates system in global system
Coordinates System ID:   0
Name: Structural Mesh Fixed
Factor from length unit to meter:    0.10000000E+01
Factor from mass unit to Kilogram:    0.10000000E+01
Tx(m) =    0.25191550E-01
Ty(m) =    0.00000000E+00
Tz(m) =   -0.15212836E+02
Rx(deg) =    0.00000000E+00
Ry(deg) =   -0.22208073E+00
Rz(deg) =    0.00000000E+00
\end{lstlisting}

It is important to put the stl file to the same reference system. It is possible inside OpenFOAM environment with the following commands:
\begin{lstlisting}[language=bash]
$ surfaceTransform -translate "(0.25191550E-01 0.00000000E+00 -0.15212836E+02)" aa.stl bb.stl
$ surfaceTransform -rollPitchYaw "(0.0 -0.22208073E+00 0.0E0)" bb.stl cc.stl
\end{lstlisting}

\item Create the CFD mesh from the obtained file.

\item Check the position of Gauss points with Paraview file located in "Visualization" directory against the CFD mesh

\item In system/controlDict file, add the following input:

\begin{lstlisting}[language=C]
    pressurePoints
    {
		type probes; 
		// Type of functionObject
		// Where to load it from (if not already in solver)
		functionObjectLibs ("libsampling.so");
                outputControl   timeStep; 
		outputInterval 1;
		fixedLocations false; // so that probes move with the mesh
		probeLocations // Locations to be probed. runTime modifiable!
		(
			#include "OpenFOAM_GaussPoints.inc"
		);
		// Fields to be probed. runTime modifiable!
		fields
		(
		p
		);

    }
\end{lstlisting}

with a reference to the coordinates of the Gauss points given by HOMER (file "*.inc").

\item Copy-paste mass properties "sixDofDomainBody" given by HOMER in 0/uniform

\item Launch foamStar computation

\item Transfer to HOMER the result of postProcessing folder, both the time history of body motion and pressure points. 

\end{itemize}





