\chapter{Pre-computing structural information}
\label{homer}

All structural information needed for mechanical equation solving and flexible modes can be pre-computed from an Homer 2 model.

\section{Running Homer}
Once the model has been set up, HmFEM has to be run with the following body options. Please refer to Homer user guide for more information about Homer keywords.

\begin{lstlisting}[language=fortran]
# Strcutural mesh information
STRUCTURALMESH = body.dat
STRUCTURALCOORDINATES = 0. 0. -11.75 0. 0. 0.
STRUCTURALLENGTHUNIT = m
STRUCTURALMASSUNIT = kg

# Number of flexible modes if required
STRUCTURALNFLEXIBLEMODES = 3

# Enable specific outputs for OpenFoam
PREPAREOPENFOAM = 1

# Sections definition
SECTIONS
CUT   -6.00 0. 12.015 1. 0. 0.
CUT   82.83 0. 12.015 1. 0. 0.
CUT  150.78 0. 12.015 1. 0. 0.
CUT  223.23 0. 12.015 1. 0. 0.
CUT  295.00 0. 12.015 1. 0. 0.
ENDSECTIONS
\end{lstlisting}

HmFEM will then run normally and create the following files that are useful for foamStar:
\begin{itemize}
\item mesh : Finite\_Elements\_Analysis{\textbackslash}HmFEM{\textbackslash}\textbf{body\_Bulk.dat}
\item mass matrix : Finite\_Elements\_Analysis{\textbackslash}HmFEM{\textbackslash}\textbf{body\_dmig.pch}
\item mode shapes : Finite\_Elements\_Analysis{\textbackslash}HmFEM{\textbackslash}\textbf{body\_md.pch}
\item don file : User\_Outputs{\textbackslash}HmFEM{\textbackslash}\textbf{body.don}
\end{itemize}

\section{Initialize structural information for foamStar}

Structural data pre-processing for foamStar is performed using the module \emph{initFlx} together with a standard OpenFoam input file \emph{initFlxDict} described hereafter:

\begin{lstlisting}[language=C]
FEM_STRUCTURALMESH_VTU
{
    datFile "body_Bulk.dat";
    mdFile "body_md.pch"; selected (7 8 9);
    dmigMfile "body_dmig.pch";
    dmigKfile "body_dmig.pch";
    pchCoordinate (0 0 -11.75 0 0 0);
    pchScaleMode  0.25909281809220411E+05;
    pchLengthUnit 1;
    pchMassUnit   1;

    patches (ship); ySym (true);
}
\end{lstlisting}

\begin{itemize}
\item \textbf{datFile} : finite element mesh provided by Homer (*\_Bulk.dat)
\item \textbf{mdFile} : modes shapes punch file given by Homer (*\_md.pch)
\item \textbf{selected} : select flexible modes to use for simulation (starting from mode 7)
\item \textbf{dmigMfile} : mass matrix punch file provided by Homer (*\_dmig.pch)
\item \textbf{dmigKfile} : stiffness matrix punch file provided by Homer (*\_dmig.pch)
\item \textbf{pchCoordinate} : structural coordinates used in Homer
\item \textbf{pchScaleMode} : value of scaling factor of first mode in Homer
\item \textbf{pchLengthUnit} : length scaling factor to meters (m=1, mm=0.001)
\item \textbf{pchMassUnit} : mass scaling factor to kilograms (kg=1, t=1000)
\item \textbf{pchMassUnit} : mass scaling factor to kilograms (kg=1, t=1000)
\item \textbf{patches} : define patch representing the hull
\item \textbf{patches} : define if Y-symmetry is used or not
\end{itemize}

foamStar module can then be launched with the following command:
\begin{lstlisting}[language=bash]
$ initFlx
\end{lstlisting}




